\documentclass[a4paper]{article}
\usepackage[utf8]{inputenc}
\usepackage[backend=biber, style=authoryear]{biblatex}
\addbibresource{references.bib}

% Custom commands
\newcommand{\delO}{\ensuremath{\delta ^{18}}O}
\begin{document}

\title{How important is ocean circulation in abrupt change of the West African Monsoon}
\author{Chad Stainbank}
\maketitle

From June to September, West Africa is inundated by a band of monsoon rainfall, originating at the Gulf of Guinea, that reaches a peak in intensity and latitudinal extent in August \parencite{sultan2003west}.
This \emph{West African Monsoon} (WAM) is of immense importance to the inhabitants of the Sahel, the semi-arid belt marking its northernmost zone of influence.
Through the 1970s and 1980s, the region's heavily agricultural economy and society were greatly impacted by ecological and environmental degradation associated with extended periods of drought \parencite{benson1998impact, olsson1993causes, walther2016review}.
These anomalously dry conditions initially provoked concern that a theorised biogeophysical feedback --- in which the rising albedo of a devegetated land surface enhances the sinking of cold, aridifying air \parencite{charney1975dynamics, charney1975drought} --- would drive a rapid southward expansion of the Sahara desert that would devastate the region.
Ultimately this runaway desertification did not come to pass; satellite observations instead indicate a widespread \emph{greening} of the Sahel in recent decades \parencite{olsson2005recent, dardel2014re}.

Instinctively, one would attribute this revegetation to the tentative recovery in rainfall over the Sahel observed over the same period \parencite{lebel2009recent}.
However, mean precipitation trends conceal the high spatial variability in rainfall patterns as well as a shift toward infrequent, extreme events \parencite{nicholson2013west, panthou2014recent}, while there remain considerable methodological concerns in the extraction and interpretation of various vegetation indices from remote sensing data \parencite{fensholt2013assessing, dardel2014rain}.
% In addition, there is increasing recognition that the traditional mechanisms by which modulation of the WAM's characteristics have been explained are inaccurate \parencite{nicholson2013west}, as discussed later
As these issues hamper efforts to discern the precise nature of the relationship between monsoon rainfall and land surface conditions, especially at local scales, it is difficult to make predictions of the response of the Sahellian landscape to future changes in global climate.
This is further complicated by the prospect of a threshold-like response to large scale changes in the WAM indicated in the North African palaeoenvironmental record.

% Evidence of existence of green Sahara and rapid state change

There are a wide range of proxy data indicating that North Africa was highly arid during the Last Glacial Maximum \parencite{harrison2001role}, including peaks in the rate of aeolian dust deposition recorded in cores of polar ice \parencite{petit1990palaeoclimatological} and marine sediment \parencite{tiedemann1989climatic, rea1994paleoclimatic}. 
However, the presence of thick layers of pollen-laden lacustrine sediments in presently hyperarid areas attests to a transition to a humid climatology, able to support extensive and highly productive wetland ecosystems, over the subsequent deglaciation \parencite{ritchie1985sediment, lezine1990across}.
On the assumption that the balance between water input and evaporation controls the oxygen isotopic signal in lake sediments, \cite{gasse1990arid} find that high water levels succeeded arid conditions in the Sahara from the $\sim$15 ka termination Heinrich Stadial to $\sim$5 ka, with a brief positive \delO{} excursion signalling the resumption of aridity during the Younger Dryas stadial (YD).
In addition, satellite radar imagery has been used to identify the geomorphological imprint of immense Saharan lakes which attest to the magnitude of this \emph{African Humid Period} (AHP) \parencite{schuster2005holocene, drake2006shorelines}.
Although highly local influences such as topography and groundwater availability can complicate extrapolation to a regional palaeclimate \parencite{baumhauer1991palaeolakes}, these proxy data are generally accepted to provide strong evidence for an extreme northward incursion of the WAM into North Africa during the mid-Holocene.

\printbibliography{}

\end{document}
