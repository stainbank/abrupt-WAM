\documentclass[a4paper]{article}
\usepackage[utf8]{inputenc}
\usepackage[backend=biber, style=nature]{biblatex}
\addbibresource{references.bib}
\linespread{1.5}
\usepackage[none]{hyphenat}
\usepackage{nopageno}

% Custom commands
\newcommand{\delO}{\ensuremath{\delta ^{18}}O}
\begin{document}

\title{How important is ocean circulation in abrupt change of the West African Monsoon}
\author{Chad Stainbank}
\maketitle

\section{West African Monsoon}
\label{sec:sahelwam}
The West African Monsoon (WAM), sweeps northeastwards across Africa from the Gulf of Guinea in June before reaching peaking over the Sahel, the continent-wide belt of land bordering the Sahara desert, in August \parencite{sultan2003west2, nicholson2013west}.
This semi-arid region marks the latitudinal limit of the monsoon's climatic influence, and the amount and distribution of seasonal rainfall is immensely important to it's heavily agriculture-dependent economy and rapidly growing population \parencite{sissoko2011agriculture}.
During the 1970--80s, great swathes of the Sahel were afflicted by ecological and environmental degradation associated with extended periods of drought, resulting in widespread crop failures which threatened millions with famine \parencite{benson1998impact, olsson1993causes, walther2016review}.
The severe long-term impact of this drought underscores why the Sahel is identified as a climate change vulnerability "hotspot" \parencite{diffenbaugh2012climate}, with the latest IPCC Assesment Report \parencite{ipcc2014impacts} asserting that extensive ecosystem loss and increases in food and water insecurities are highly likely.
The proactive adaptation to anticipated precipitation trends is therefore a regional priority \parencite{lobell2008prioritizing, sarr2012present}.

The emission of mineral dust into the atmosphere from the Sahara and the Sahel is supressed by monsoon precipitation and the vegetation cover it sustains \parencite{brooks2000dust, cowie2013vegetation}.
Terrigenous aerosols variably absorb and scatter radiation in both directions \parencite{andreae1995climatic, tegen1996influence, haywood2000estimates} and alter both local and remote clouds in terms of formation and thermodynamic properties \parencite{rosenfeld2001desert, demott2003african, huang2006possible, karydis2011effect}.  
Therefore, as the dominant global dust source \parencite{tanaka2006numerical}, North Africa has a major influence on the Earth's radiation budget, although uncertainties in the magnitude, vertical distribution profile, and chemical composition of this atmospheric particulate load severely hamper destermination of even the sign of the net radiative forcing \parencite{claquin1998uncertainties, sokolik2001introduction, durant2009sensitivity}.

Saharan dust deposited over nearby oceans provides an important supply of iron, nitrogen and phosphorus to marine ecosystems \parencite{prospero1996saharan, prospero1996atmospheric, guieu2002chemical, mills2004iron, bristow2010fertilizing, okin2011impacts}.
This fertilisation effect is perhaps more significant in the Amazon Basin --- a carbon sink in decline \parencite{brienen2015long} --- where the intercontinental nutrient flux is estimated to entirely replenish rainforest soils of continually rain-stripped phosphorous \parencite{swap1992saharan, bristow2010fertilizing, yu2015fertilizing}.
With dust an integral feature of the global biogeochemical cycle \parencite{ridgwell2002dust, harrison2001role, jickells2005global, mahowald2005atmospheric}, the concentration of its emission in a few ``hotspot'' sources \parencite{koren2006bodele, ben2010transport, schepanski2009saharan, knippertz2010central} means that global biosphere productivity may be highly sensitive to small changes in the North African land surface, making the WAM a potential driver of abrupt future change in the climate system.

\section{Historic WAM}
Analyses of North African climate trends provide a foundation for the evaluation of the potential WAM response to rising CO\textsubscript{2} \parencite{redelsperger2006african}, and satellite observations reveal widespread \emph{greening} of the Sahel in recent decades \parencite{olsson2005recent, dardel2014re}.
However, methodological issues in the extraction and interpretation of vegetation indices from remote sensing data make land-surface conditions an unreliable indicator of regional climate \parencite{fensholt2013assessing, dardel2014rain}.
Among direct rainfall studies, there remains considerable disagreement regarding the extent of recovery from late twentieth century drought conditions \parencite{nicholson2005question, nicholson2013west}.
Furthermore, the mean Sahellian climate trend conceals highly spatially discontinuous underlying changes, such as increasing interannual variability and a shift in the toward a more extreme Central Sahel rainfall regime \parencite{nicholson2013west, lebel2009recent, panthou2014recent}, so this statistic cannot be simply extrapolated to the future.
An improved understanding of the mechanisms governing spatiotemporal variability of the WAM is therefore required to inform regional climate projections \parencite{redelsperger2006african}.

A strong correlation between North African rainfall patterns and global sea surface temperature (SST) anomalies has long been indicated by a range of observational and modelling studies, for which \citeauthor{rodriguez2015variability} \parencite{rodriguez2015variability} provide an excellent overview.
Briefly, a more southward (northward) monsoon position, with weaker (stronger) precipitation over the Sahel, appears to be driven by warmer (colder) SSTs in the tropical eastern Pacific \parencite{rowell1995variability, fontaine1996sea, rowell2001teleconnections, janicot2001summer, giannini2003oceanic} and Indian \parencite{rowell1995variability, fontaine1996sea, bader2003impact, giannini2003oceanic} oceans --- revealing the fingerprint of the El Ni\~{n}o Southern Oscillation (ENSO) on African climate \parencite{janowiak1988investigation, rowell1995variability, ward1998diagnosis} --- while Sahel rainfall is distinctly enhanced by Mediterranean warming \parencite{rowell2003impact, gaetani2010influence, fontaine2010impacts}.
The influence of Atlantic variability is more complex: drought is intensified by increased Gulf of Guinea SSTs during peak years of the Atlantic Ni\~{n}o \parencite{lamb1978case, bah1987towards, rowell1995variability, fontaine1996sea, ward1998diagnosis, giannini2003oceanic}, whereas low frequency analyses show positive correlation between rainfall and warm phases of the Atlantic Multidecadal Oscillation (AMO) \parencite{zhang2006impact, martin2014impact}.
The latter relationship appears integral to the strong climate variability forcing of hemispherically asymmetric SST changes \parencite{folland1986sahel, lamb1978large, rowell1995variability, fontaine1998evolution, ward1998diagnosis}, with one study \parencite{hoerling2006detection} attributing the late twentieth century Sahellian drought primarily to the relative warming of the tropical South Atlantic. 

General Circulation Model (GCM) experiments allow the exploration of the North African rainfall response to future climate change, and the mechanistic interpretation of such simulations is largely informed by the \emph{classical} model of WAM dynamics.
Briefly \parencite[see][for comprehensive descriptions]{sultan2000abrupt, sultan2003west2, ramel2006northward}, it posits that the Monsoon flow, a low-level southwesterly flow of marine moisture driven cyclonically over the African continent by the summer surface pressure gradient between the intense Saharan Heat Low and the cooler Atlantic \parencite{hall2006dynamics, grams2010atlantic}, is precipitated by deep convection at the Intertropical Convergence Zone (ITCZ), the subsolar point--chasing band of trade wind confluence representing the rising branch of the Hadley circulation.
The theory holds that WAM characteristics are primarily determined by the surface moisture flow to \parencite{folland1986sahel, sultan2003west1, hagos2007dynamics}, and strength and latitudinal position of \parencite{janicot1998west, damato1998characteristics, sultan2003west1}, the continental ITCZ --- with which the band of precipitation maxima is often treated synonymously \parencite[e.g.][]{shinoda1994tropical, ba1995satellite, ramel2006northward, braconnot2007results, peyrille2016annual}.

Accordingly, the ENSO-Sahel teleconnection is attributed to the generation, by convective anomalies over the warm Indo-Pacific basin, of westward-propogating equatorial Rossby and Kelvin waves which interact to induce large-scale mid-tropospheric subsidence over the Sahel, thereby weakening the cyclonic circulation of the Monsoon flow and suppresing ITCZ convection \parencite{shinoda1994tropical, goddard1999importance, rowell2001teleconnections, janicot2001intra, lu2005oceanic}.
Raised equatorial Atlantic SSTs flatten the land--ocean thermal contrast, resulting in less continentward Monsoon flow penetration and reduced moisture convergence \parencite{vizy2002development, giannini2003oceanic, losada2010multi}, although an enhanced northerly supply of moisture evaporated from a warming Mediterranean provides a counteractive effect \parencite{rowell2003impact, gaetani2010influence}.
A weakening of the low-level surface monsoon circulation may also result from south-biased ocean warming, particurly in the tropical Atlantic, via a reduction of the cross-equatorial atmospheric pressure gradient \parencite{lu2005oceanic, chung2007relationship, biasutti2006robust, zhang2006impact}, although an alternative mechanism proposes that an equatorword ITCZ shift represents the organisational response of the Hadley circulation to the meridional energy imbalance effected by relative cooling of the extra-tropical North Atlantic \parencite{broccoli2006response, hwang2013anthropogenic}. 

Recent years have seen the emergence of a new paradigm of WAM dynamics in which monsoon rainfall is not dependent on the aforedescribed surface circulation \parencite{grist2001study} but generated instead by the interaction of zonal jets higher in the troposphere \parencite[see][for a comprehensive overview]{nicholson2009revised}.
This \emph{revised} framework defines the ``tropical rainbelt'' as a column of deep convection maintained between the axes of the African Easterly Jet (AEJ), a mid-level quasi-geostrophic wind, and the high-level Tropical Easterly Jet (TEJ).
The rainbelt's northward extent is bound to the position of the mobile AEJ, while its uplift strength is modulated by the speed of the TEJ at it's southern edge \parencite{gu2004seasonal, nicholson2009revised, nicholson2013west}.
With regards to moisture, a highly variable supply to the Sahel is provided from the eastern Atlantic low-level westerly jets, regional-scale circulation features recently identified as distinct from, and dominant over, the canonical Monsoon flow \parencite{grodsky2003near, pu2010dynamics, nicholson2013west}.

While Sahel rainfall hindcast studies assert the importance of simulating these atmospheric features \parencite{philippon2010skill, xue2010intercomparison, ruti2011west, diallo2013interannual}, the identification of relevant mechanisms of oceanic influence remains challenging as typical GCM spatiotemporal resolutions are too coarse for sufficiently realistic representions of jet dynamics \parencite{caminade2010twentieth, druyan2011studies, tseng2016diagnosing, vellinga2016sahel, whittleston2017climate}.
In the future, increasing tropical SSTs will stabilise the lower troposphere, raising the moisture threshold required to trigger deep convective rainfall (``upped-ante'' mechanism) \parencite{neelin2003tropical, caminade2010twentieth, liu2014atmospheric}. 
This may be counteracted by the rainfall-inducing changes in upper atmosphere circulation associated with relative warming of the extratropical North Atlantic \parencite{liu2014atmospheric, martin2014impact, park2015northern, monerie2016range}, although no study has yet mechanistically described this teleconnection.
It has been proposed that El Ni\~{n}o events move the AEJ equatorward, and weaken the TEJ and the westerly jets, though the establishment of anomalous equatoreal zonal circulation \parencite{joly2009influence, okonkwo2015combined, villamayor2015robust, preethi2015impacts}.
Furthermore, westerly moisture flow strength and AEJ position are both sensitive to the cross-equatorial thermal gradient over West Africa, which is in turn determined by land surface conditions, such that an initial pertubation to soil moisture or vegetation cover in the Sahel might become rapidly self-reinforcing \parencite{cook1999generation, thorncroft1999maintenance, patricola2008atmosphere, pu2012role, liu2014atmospheric, berg2017soil}.

The conflict between the two interpretations of WAM dynamics underscores the scope for development in the mechanistic understanding of historic North African rainfall variability, particularly with regard to the relative impacts of different ocean basins \parencite{lafore2011progress, roehrig2013present, nicholson2013west, rodriguez2015variability, martin2016understanding}.
As a result, uncertainty in GCM projections of Sahel climatology is dominated by intermodel variability in the atmospheric response to prescribed SST forcings \parencite{cook2006coupled, druyan2011studies, roehrig2013present, lee2014future, chadwick2016aspects}, while the degree of amplification by land-surface feedbacks remains poorly quantified \parencite{nicholson2000land, taylor2011new}. 
Furthermore, projection uncertainty is not reduced upon selection of models by hindcast skill \parencite{rowell2016can}, which suggests that the instrumental record is an inadequate constraint on the future rainfall in North Africa.
Past WAM variability, as recorded in palaeoenvironmental data, must therefore be used to inform and assess projections of potential change \parencite{braconnot2012evaluation, mohtadi2016palaeoclimatic}.

\section{The palaeoenvironmental record of the WAM}

Alternating arid--humid phases in North Africa, paced by Milankovitch cycles and the associated Quaternary glaciations, can be discerned through the use of marine core sediment's magnetic properties \parencite{bloemendal1989evidence, larrasoana2003three} and non-carbonate fraction \parencite{tiedemann1989climatic, tiedemann1994astronomic} as proxies for Saharan dust emission.
Most recently, Last Glacial Maximum (LGM) hyperaridity gave way to a mid-Holocene dust minimum known as the African Humid Period (AHP) \parencite{rea1994paleoclimatic, demenocal2000abrupt, adkins2006african}.
This stands in agreement with continental evidence that, until \textasciitilde 5 ka, the regional climatology supported extensive, highly productive wetland ecosystems, including the geomorphological imprints of immense Sarahan lakes apparent in satellite radar imagery \parencite{schuster2005holocene, drake2006shorelines} and palaeohydroecological reconstructions from thick layers of fossil pollen-laden lacustrine sediment \parencite{ritchie1985sediment, lezine1990across, jolly1998biome}. 
Although spatially variable factors complicate the extrapolation of local palaeoclimate indicators \parencite{baumhauer1991palaeolakes}, these proxy data are generally accepted to demonstrate an extreme northward translocation of the WAM relative to modern day.

The last glacial-interglacial climatalogical transition was not smooth: high resolution Atlantic palaeosedimentation records indicate a rapid onset of the AHP as the Heinrich Stadial 1 (HS1) gave way to the Bølling–Allerød, an abrupt spike in the Saharan dust flux during the Younger Dryas stadial (YD), and, finally, a century-scale return to modern arid conditions following the Holocene Climatic Optimum (HCO) \parencite{demenocal2000abrupt, kuhlmann2004transition, adkins2006african, mcgee2013magnitude, ehrmann2013dynamics, collins2013abrupt, williams2016glacial}.
This suggests that the drastic, ice sheet-forced cessation of Atlantic Meridional Overturning Circulation (AMOC) associated with the late Pleistocene stadials \parencite{mcmanus2004collapse, lynch2017atlantic, ritz2013estimated} may have, by the modification of the global SST pattern \parencite{boyle1987north, kiefer2005patterns, kienast2006eastern, barker2009interhemispheric}, driven the abrupt drying events recorded in North Africa \parencite{weldeab2007155, mulitza2008sahel, collins2013abrupt}. 
However, terrigenous marine sedimentation may be a misleading land surface proxy, considering the spatial heterogeneity in dust emission, the strong influence of wind strength on the oceanward aeolian flux \parencite{ruddiman1997tropical, mcgee2010gustiness, parker2016new}, and the post-depositional mixing issues inherent to sediment records \parencite{mahowald1999dust, giresse2003late, maslin2003evidence}.
Nevertheless, the synchrony of abrupt transitions with large-scale reorganisations of ocean circulation is corroborated by onshore evidence \parencite{gasse1994abrupt, garcin2007abrupt, talbot2007abrupt}, including a brief positive oxygen isotope (\delO) excursion in palaeolake carbonates signalling the interruption of high water levels by YD aridity \parencite{gasse1990arid}.
By contrast, palaeoenvironmental evidence for an abrupt end to AHP \parencite{gasse1990arid, salzmann2005dahomey, tierney2013abrupt, tierney2017rainfall} is contradicted by palynological vegetation reconstructions presenting a more progressive land surface response to gradually weakening seasonal insolation \parencite{kropelin2008climate, lezine2009timing, vincens2010vegetation, amaral2013palynological, shanahan2015time}. 

Although transient simulation experiments assert the requirement of AMOC state switching for the \emph{rapidity} of North African transitions at H1 and YD to be reproduced, the resultant changes are severely underestimated in \emph{magnitude} \parencite{timm2010mechanisms, otto2014coherent}. 
Likewise, none of the GCMs in the second Paleoclimate Modelling Intercomparison Project (PMIP2) are capable of approaching observations of HCO precipitation over the Sahel when run to equilibrium with 6 ka insolation, with only modest improvement offered by the inclusion of explicit vegetation dynamics \parencite{braconnot2012evaluation, zheng2013characterization, harrison2015evaluation}.
This systematic underestimation of WAM sensitivity to ice sheet and orbital forcings is indicative of structural deficiencies in the current generation of GCMs, such as the substanial underestimation, or even omission, of various processess which could amplify vegetation change \parencite{harrison2015evaluation, tierney2017rainfall}.
However, it is difficult to improve the characterisation of these land-surface--atmosphere feedbacks in GCMs while the question of threshold behaviour at the AHP termination remains unresolved \parencite{claussen2017theory}. 
Another proposed source of serious model bias is the muted response of the meridional temperature gradient over West Africa to altered insolation \parencite{zheng2013characterization}, although the competing interpretations of modern WAM dynamics severely complicates the assessment of it's significance. 
Nevertheless, advancements in GCMs and the development of region-wide palaeodata syntheses should bring about a reduction in these AHP model--data discrepancies by the next PMIP phase, thereby allowing projections of future change in the Sahel to be improved \parencite{braconnot2012evaluation, harrison2015evaluation}.

\printbibliography{}

\end{document}
