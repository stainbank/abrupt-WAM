\documentclass[a4paper]{article}
\usepackage[utf8]{inputenc}
\usepackage[backend=biber, style=nature]{biblatex}
\addbibresource{references.bib}

% Custom commands
\newcommand{\delO}{\ensuremath{\delta ^{18}}O}
\begin{document}

\title{How important is ocean circulation in abrupt change of the West African Monsoon}
\author{Chad Stainbank}
\maketitle

\section{North Africa and the global climate system}
\label{sec:sahelwam}
During a typical August, the West African Monsoon (WAM), having swept from the Gulf of Guinea in June northeastwards across Africa, will reach a peak in intensity over the Sahel, the continent-wide belt of land bordering the Sahara desert that marks the latitudinal limit of its climatic influence \parencite{sultan2003west2, nicholson2013west}.
Given that this semi-arid region has a rapidly growing population that is heavily dependent on agriculture \parencite{sissoko2011agriculture}, the amount and distribution of monsoon rainfall is of immense societal and economic importance.
Indeed, during the 1970s and 1980s, great swathes of the Sahel were afflicted by ecological and environmental degradation associated with extended periods of drought, resulting in widespread crop failures which threatened millions of people with famine \parencite{benson1998impact, olsson1993causes, walther2016review}.
Although initial fears of runaway desertification \parencite{charney1975dynamics, charney1975drought} did not come to pass, the severe long-term impact of this drought underscores the sensitivity of the region to a changing climate.
The latest IPCC Assesment Report \parencite{ipcc2014impacts} outlines the numerous climate vulnerabilities of North Africa, including a high likelihood of extensive ecosystem loss and increasing food and water insecurity, while the Sahel is identified as a climate change "hotspot" \parencite{diffenbaugh2012climate}.
Proactive adaptation to expected rainfall trends is, therefore, a regional priority \parencite{lobell2008prioritizing, sarr2012present}.

Monsoon precipitation, and the vegetation cover it sustains, has a suppressive effect on the quantity of mineral dust emitted into the atmosphere from the Sahara and the Sahel \parencite{brooks2000dust, cowie2013vegetation}.
Terrigenous aerosols act to both absorb and scatter incoming and outgoing radiation \parencite{andreae1995climatic, tegen1996influence, haywood2000estimates}, as well as to enhance and suppress precipitative cloud formation and alter cloud properties, locally and further afield \parencite{rosenfeld2001desert, demott2003african, huang2006possible, karydis2011effect}  
As the dominant global dust source \parencite{tanaka2006numerical}, North Africa plays a major role in determining the Earth's radiation budget, although uncertainty in the magnitude, vertical profile, size distribution and chemical composition of this atmospheric particulate load severely hamper estimates of the size, or even sign, of the net radiative forcing \parencite{claquin1998uncertainties, sokolik2001introduction, durant2009sensitivity}.

The deposition of desert dust over the North Atlantic and Mediterranean provides an important supply of iron, nitrogen and phosphorus to marine ecosystems \parencite{prospero1996saharan, prospero1996atmospheric, guieu2002chemical, mills2004iron, bristow2010fertilizing, okin2011impacts}.
This fertilisation effect is even more significant in the Amazon Basin --- whose carbon sink is in decline \parencite{brienen2015long} --- as Saharan dust is estimated to entirely replenish rainforest soils of the soluble phosphorous continually denuded by rainfall \parencite{swap1992saharan, bristow2010fertilizing, yu2015fertilizing}.
Since this intercontinental nutrient supply is dominated by a few dust source "hotspots" \parencite{koren2006bodele, ben2010transport, schepanski2009saharan, knippertz2010central}, the productivity of nearby oceans and of the Amazon rainforest may be sensitive to even small changes in climate conditions in the Sahara and the Sahel.
With dust an integral feature of the global biogeochemical cycle \parencite{ridgwell2002dust, harrison2001role, jickells2005global, mahowald2005atmospheric}, the North African land surface may be key driver of future change in the climate system.
low-levellow-level  
\section{WAM Variability}
Analyses of recent North African climate trends provide a foundation for the evaluation of the potential response of the WAM to rising CO\textsubscript{2} \parencite{redelsperger2006african}. 
Satellite observations indicate that the Sahel has undergone widespread \emph{greening} in recent decades \parencite{olsson2005recent, dardel2014re}, although there are methodological issues in the extraction and interpretation of vegetation indices from remote sensing data that make land-surface conditions an unreliable indicator of regional climate \parencite{fensholt2013assessing, dardel2014rain}.
Among direct rainfall studies, there remains considerable disagreement as to the extent of recovery from the drought conditions of the late twentieth century \parencite{nicholson2005question, nicholson2013west}.
Furthermore, consideration of the mean Sahellian climate trend conceals the highly spatially discontinuous underlying changes, such as increasing interannual variability and a shift in the Central Sahel toward a more extreme rainfall regime \parencite{nicholson2013west, lebel2009recent, panthou2014recent}.
Since it is not appropriate to simply extrapolate rainfall statistics, a thorough understanding of the mechanisms which govern spatiotemporal variability of the WAM is required to inform North African climate projections \parencite{redelsperger2006african}.

A strong correlation between North African rainfall patterns and global sea surface temperature (SST) anomalies has long been indicated by a range of observational and modelling studies, for which \citeauthor{rodriguez2015variability} \parencite{rodriguez2015variability} provide an excellent overview. % }. 
Briefly, a more southward (northward) monsoon position, with weaker (stronger) precipitation over the Sahel, appears to be driven by warmer (colder) SSTs in tropical eastern Pacific \parencite{rowell1995variability, fontaine1996sea, rowell2001teleconnections, janicot2001summer, giannini2003oceanic} and Indian \parencite{rowell1995variability, fontaine1996sea, bader2003impact, giannini2003oceanic} oceans, revealing the fingerprint of the El Ni\~{n}o Southern Oscillation (ENSO) on African climate \parencite{janowiak1988investigation, rowell1995variability, ward1998diagnosis}.
Warming of the Mediterranean Sea has a clear positive impact on rainfall in the Sahel \parencite{rowell2003impact, gaetani2010influence, fontaine2010impacts}, while the influence of Atlantic variability is more complex.
Drought is intensified by increased SSTs in the Gulf of Guinea during peak years of the Atlantic Ni\~{n}o \parencite{lamb1978case, bah1987towards, rowell1995variability, fontaine1996sea, ward1998diagnosis, giannini2003oceanic}, whereas low frequency analyses show positive correlation between rainfall and warm phases of the Atlantic Multidecadal Oscillation (AMO) \parencite{zhang2006impact, martin2014impact}.
The latter relationship appears to be integral to the strong climate variability forcing of differential hemispheric SST changes \parencite{folland1986sahel, lamb1978large, rowell1995variability, fontaine1998evolution, ward1998diagnosis}, with \citeauthor{hoerling2006detection} \parencite{hoerling2006detection} attributing the late twentieth century Sahellian drought primarily to the warming of the tropical South Atlantic relative to tropical North. 

General Circulation Model (GCM) experiments allow the exploration of the North African rainfall response to future climate change, and the mechanistic interpretation of such simulations is largely informed by the \emph{classical} model of WAM dynamics.
Briefly \parencite[see][for comprehensive descriptions]{sultan2000abrupt, sultan2003west2, ramel2006northward}, it posits that the Monsoon flow --- a low-level southwesterly flow of marine moisture driven cyclonically over the African continent by the surface pressure gradient between the Saharan Heat Low (SHL), an intense thermal low that develops during the boreal summer over the Northern Mauritania desert \parencite{engelstaedter2015saharan}, and the cooler Atlantic Ocean \parencite{hall2006dynamics, grams2010atlantic} --- is precipitated by deep convection at the Intertropical Convergence Zone (ITCZ), the subsolar point--chasing band of trade wind confluence that represents the rising branch of the Hadley circulation.
This theory holds that the characteristics of the WAM are determined primarily by the surface moisture flow to \parencite{folland1986sahel, sultan2003west1, hagos2007dynamics}, and strength and latitudinal position of \parencite{janicot1998west, damato1998characteristics, sultan2003west1}, the continental ITCZ --- with which the band of precipitation maxima is often treated synonymously \parencite[e.g.][]{shinoda1994tropical, ba1995satellite, ramel2006northward, braconnot2007results, peyrille2016annual}.

Accordingly, the ENSO-Sahel teleconnection is attributed to the generation, by convective anomalies over the warm Indo-Pacific basin, of equatorial Rossby and Kelvin waves which propogate westwards and interact over the Sahel to induce large-scale mid-tropospheric subsidence, resulting in the reduction of seasonal rainfall by weakening the cyclonic circulation of the Monsoon flow and suppresion of ITCZ convection \parencite{shinoda1994tropical, goddard1999importance, rowell2001teleconnections, janicot2001intra, lu2005oceanic}.
Raised equatorial Atlantic SSTs lessen the land--ocean thermal contrast, resulting in less continentward Monsoon flow penetration and reduced moisture convergence over the Sahel \parencite{vizy2002development, giannini2003oceanic, losada2010multi}, although an increased northerly supply of moisture evaporated from a warming Mediterranean provides a counteractive effect \parencite{rowell2003impact, gaetani2010influence}.
This weakening of the low-level surface monsoon circulation has also been used to explain the impact of a more south-biased interhemispheric SST gradient, particurly in the tropical Atlantic, via a reduction in the cross-equatorial atmopsheric pressure gradient \parencite{lu2005oceanic, chung2007relationship, biasutti2006robust, zhang2006impact}, although an alternative mechanism proposes that an equatorword ITCZ shift represents the organisational response of the Hadley circulation to the meridional energy imbalance induced by relative cooling of the extra-tropical North Atlantic \parencite{broccoli2006response, hwang2013anthropogenic}. 

However, recent years have seen the emergence of a new paradigm of WAM dynamics in which monsoon rainfall is largely independent of the aforedescribed surface circulation \parencite{grist2001study}; it is generated instead by the interaction of zonal jets higher in the troposphere \parencite[see][for a comprehensive overview]{nicholson2009revised}.
This \emph{revised} framework defines the ``tropical rainbelt'' as a column of deep convection maintained between the axes of the African Easterly Jet (AEJ), a mid-level quasi-geostrophic wind, and the high-level Tropical Easterly Jet (TEJ).
The rainbelt's position is bound to that of the mobile AEJ to the north, while its uplift strength is modulated by the speed of the more static TEJ at it's southern edge \parencite{gu2004seasonal, nicholson2009revised, nicholson2013west}.
With regards to moisture, a highly variable supply to the Sahel is provided from the eastern Atlantic low-level westerly jets, regional-scale circulation features recently identified as distinct from, and dominant over, the canonical Monsoon flow \parencite{grodsky2003near, pu2010dynamics, nicholson2013west}.

While recent Sahel rainfall hindcast studies assert the importance of simulating these atmospheric features \parencite{philippon2010skill, xue2010intercomparison, ruti2011west, diallo2013interannual}, it remains difficult to identify the relevant mechanisms of oceanic influence since GCM spatiotemporal resolutions are, generally speaking, too coarse for a sufficiently realistic represention of jet dynamics \parencite{caminade2010twentieth, druyan2011studies, tseng2016diagnosing, vellinga2016sahel, whittleston2017climate}.
In the future, increasing tropical SSTs will stabilise the lower troposphere, raising the moisture threshold required to trigger deep convection (``upped-ante'' mechanism) \parencite{neelin2003tropical, caminade2010twentieth, liu2014atmospheric}. 
This may be counteracted by the rainfall-inducing changes in upper atmosphere circulation associated with relative warming of the extratropical North Atlantic \parencite{liu2014atmospheric, martin2014impact, park2015northern, monerie2016range}, although no study has yet mechanistically described this teleconnection.
It has been proposed that El Ni\~{n}o events move the AEJ equatorward, and weaken the TEJ and the westerly jets, though the establishment of anomalous equatoreal zonal circulation \parencite{joly2009influence, okonkwo2015combined, villamayor2015robust, preethi2015impacts}.
Furthermore, the strength of the westerly moisture flow and the position of the AEJ are both sensitive to the cross-equatorial thermal gradient over West Africa, which is in turn determined by land surface conditions, such that any initial change in soil moisture and vegetation cover in the Sahel can become self-reinforcing \parencite{cook1999generation, thorncroft1999maintenance, patricola2008atmosphere, pu2012role, liu2014atmospheric, berg2017soil}.

%% Summary of issues, final para in this section
% These two competing interpretations are both \parencite{lafore2011progress, peyrille2016annual, see post 2009 wam dynamics articles.
% Different competing mechs, future depends on relative strengths of oceanic basins 
% There is no agreement with regard to the climatological future of the Sahel \parencite{druyan2011studies}.
% Many studies projections of future are validated by recent, yet future changes are likely to be large compared to decadal variability.
% 
\ldots It is therefore useful to look at the palaeoenvironmental record in order to inform projections of how the North African landscape may respond to future changes in global climate \parencite{mohtadi2016palaeoclimatic}.

\section{The palaeoenvironmental record of the WAM}

Alternating arid--humid phases in North Africa, paced by Milankovitch forcing and the associated Quaternary glaciations, can be discerned through the use of marine core sediment's magnetic properties \parencite{bloemendal1989evidence, larrasoana2003three} and non-carbonate fraction \parencite{tiedemann1989climatic, tiedemann1994astronomic} as proxies for Saharan dust emission.
Most recently, Last Glacial Maximum (LGM) hyperaridity gave way to a mid-Holocene dust minimum known as the African Humid Period (AHP) \parencite{rea1994paleoclimatic, demenocal2000abrupt, adkins2006african}.
This is in agreement with continental evidence that, until \textasciitilde 5 ka, the regional climatology supported extensive, highly productive wetland ecosystems, including the geomorphological imprints of immense Sarahan lakes apparent in satellite radar imagery \parencite{schuster2005holocene, drake2006shorelines} and palaeohydroecological reconstructions from thick layers of lacustrine sediment laden with fossil pollen \parencite{ritchie1985sediment, lezine1990across, jolly1998biome}. 
Although spatially variable factors complicate the extrapolation of local palaeoclimate indicators \parencite{baumhauer1991palaeolakes}, these proxy data are generally accepted to demonstrate an extreme northward translocation of the WAM relative to modern day.

The postglacial climatalogical transition was not smooth: high resolution Atlantic palaeosedimentation records indicate a rapid onset of the AHP following Heinrich Stadial 1 (HS1), an abrupt spike in the Saharan dust flux during the Younger Dryas stadial (YD), and a century-scale return to modern arid conditions \parencite{demenocal2000abrupt, kuhlmann2004transition, adkins2006african, mcgee2013magnitude, ehrmann2013dynamics, williams2016glacial}.
Terrigenous marine sedimentation is an imperfect proxy for Saharan land surface conditions, considering the spatial heterogeneity in desert dust emission, the strong influence of wind strength on the oceanward aeolian flux \parencite{ruddiman1997tropical, mcgee2010gustiness, parker2016new}, and the post-depositional mixing issues inherent to sediment records \parencite{mahowald1999dust, giresse2003late, maslin2003evidence}.
Nevertheless, there is an array of onshore evidence that appear to corrorate the abruptness of the transitions synchronous with global-postglacial reorganisations \parencite{}, including palaeolake carbonate records showing --- on the assumption of control of the oxygen isotopic signal (\delO) by the input--evaporation balance --- a brief positive \delO{} excursion signalling the interruption of high water levels by YD aridity \parencite{gasse1990arid}.
However, while some palaeoenvironmental records support abruptness in the AHP termination \parencite{gasse1990arid, salzmann2005dahomey, tierney2017rainfall}, many palynological reconstructions indicate a more progressive land surface response to gradually weakening insolation \parencite{kropelin2008climate, lezine2009timing, vincens2010vegetation, amaral2013palynological, shanahan2015time}. % check whether these are all palynological reconstructions
% Abruptness in response to reorganisation for onset and YD, but abruptness for termination shows nonlinear response to gradual insolation forcing.

Models show \ldots
% Do models show abruptness from orbital forcing alone, even with veg?
% The abrupt climatalogical response to gradual orbital forcing is indicative of the influence of large-scale postglacial reorganisation of global ocean circulation.
% general circulation models (GCMs) featuring atmosphere have largely failed to reproduce a threshold-like response 

\printbibliography{style=nature}

\end{document}
