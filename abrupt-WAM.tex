\documentclass[a4paper]{article}
\usepackage[utf8]{inputenc}
\usepackage[backend=biber, style=nature]{biblatex}
\addbibresource{references.bib}

% Custom commands
\newcommand{\delO}{\ensuremath{\delta ^{18}}O}
\begin{document}

\title{How important is ocean circulation in abrupt change of the West African Monsoon}
\author{Chad Stainbank}
\maketitle

From June to September, West Africa is inundated by a band of monsoon rainfall, originating at the Gulf of Guinea, that reaches a peak in intensity and latitudinal extent in August \parencite{sultan2003west}.
This \emph{West African Monsoon} (WAM) is of immense importance to the inhabitants of the Sahel, the semi-arid belt marking its northernmost zone of influence.
Through the 1970s and 1980s, the region's heavily agricultural economy and society were greatly impacted by ecological and environmental degradation associated with extended periods of drought \parencite{benson1998impact, olsson1993causes, walther2016review}.
These anomalously dry conditions initially provoked concern that a theorised biogeophysical feedback --- in which the rising albedo of a devegetated land surface enhances the sinking of cold, aridifying air \parencite{charney1975dynamics, charney1975drought} --- would drive a rapid southward expansion of the Sahara desert that would devastate the region.
Ultimately Charney's scenario of runaway desertification did not come to pass; satellite observations instead indicate a widespread \emph{greening} of the Sahel in recent decades \parencite{olsson2005recent, dardel2014re}.

Instinctively, one would attribute this revegetation to the tentative recovery in rainfall over the Sahel observed over the same period \parencite{lebel2009recent}.
However, methodological issues in the extraction and interpretation of various vegetation indices from remote sensing data have considerably frustrated efforts to discern the statistical relationship between monsoon rainfall and land surface conditions \parencite{fensholt2013assessing, dardel2014rain}.
Furthermore, any focus on mean precipitation trends conceals a large increase in spatial variability, as well as a pattern shift toward more extreme but infrequent events \parencite{nicholson2013west, panthou2014recent}.
Therefore, in order to make useful projections of the response of the Sahellian landscape to future changes in global climate, it is crucial to develop a better understanding of the mechanisms which govern the timing, strength and position of the monsoon over North Africa. 

\citeauthor{nicholson2013west} \parencite{nicholson2013west} provide a comprehensive overview of the wide range of factors theorised to modulate WAM variability at interannual to interdecadal timescales.
For example, several General Circulation Model (GCM) experiments have succesfully reproduced much of the sensitivity of monsoon rainfall to global sea surface temperatures apparent in the observational record \parencite[e.g.][]{rowell1995variability, giannini2003oceanic, lu2005oceanic}, while others demonstrate the importance of land-surface--atmosphere interactions in amplifying these oceanic forcings \parencite{giannini2005dynamics, kucharski2013further}.
However, no single dominant influence on the WAM has emerged as GCMs present an array of contradictory mechanisms of monsoon control, particularly with regard to the relative influence of each ocean basin... 
As a result, there is no agreement with regard to the climatological future of the Sahel \parencite{druyan2011studies}.
% In addition, future changes are huge compared to decadal variability.
It is therefore useful to look at the palaeoenvironmental record in order to inform projections of how the North African landscape may respond to future changes in global climate.
% This is further complicated by the prospect of a threshold-like response to large scale changes in the WAM indicated in the North African palaeoenvironmental record.

% Evidence of existence of green Sahara and rapid state change

There are a wide range of proxy data indicating that the global climate was highly arid during the Last Glacial Maximum \parencite{harrison2001role}, including peaks in the rate of aeolian dust deposition recorded in cores of polar ice \parencite{petit1990palaeoclimatological} and marine sediment \parencite{tiedemann1989climatic, rea1994paleoclimatic}. 
However, thick layers of pollen-laden lacustrine sediments in the Sahara desert attest to the region's transition, over the subsequent deglaciation, to humid conditions that could support extensive and highly productive wetland ecosystems \parencite{ritchie1985sediment, lezine1990across}.
On the assumption that the balance between water input and evaporation controls the oxygen isotopic signal in lake sediments, \citeauthor{gasse1990arid} \parencite{gasse1990arid} find that high water levels succeeded arid conditions in North Africa from the $\sim$15 ka termination Heinrich Stadial to $\sim$5 ka, with a brief positive \delO{} excursion signalling the resumption of aridity during the Younger Dryas stadial (YD).
In addition, satellite radar imagery has been used to identify the geomorphological imprint of immense Saharan lakes which attest to the magnitude of this \emph{African Humid Period} (AHP) \parencite{schuster2005holocene, drake2006shorelines}.
Although highly local influences such as topography and groundwater availability can complicate extrapolation to a regional palaeclimate \parencite{baumhauer1991palaeolakes}, these proxy data are generally accepted to provide strong evidence for an extreme northward incursion of the WAM into North Africa during the mid-Holocene.

The AHP was not a unique occurence: using marine sediment's magnetic properties \parencite{bloemendal1989evidence, larrasoana2003three}non-carbonate fraction \parencite{tiedemann1994astronomic} as a proxy for the Saharan dust flux, aridity in North Africa can be shown to have been paced by various components of the Milankovitch cycles over the past 5 Ma.
However, despite the gradualness of the orbital forcing, sharp discontinuities in very high resolution Atlantic palaeosedimentation records indicate that the most recent arid--humid transitions --- at the end of the LGM, into and out of the YD, and at the end of AHP --- were exceptionally abrupt \parencite{demenocal2000abrupt}.
The assertion of century-scale climatological transitions is not without criticism.
% For example... is it just wind?
% In addition, general circulation models (GCMs) featuring atmosphere have largely failed to reproduce a threshold-like response
% critcise evidence of abruptness - e.g. is it not just wind?
% show other evidence of abruptness
% Assuming evidence is legit, shows that there is some nonlinearity in the system
% Obvious candidate is monsoon


\printbibliography{style=nature}

\end{document}
