\documentclass[a4paper]{article}
\usepackage[utf8]{inputenc}
\usepackage[backend=biber, style=authoryear]{biblatex}
\addbibresource{references.bib}
\begin{document}

\title{How important is ocean circulation in abrupt change of the West African Monsoon}
\author{Chad Stainbank}
\maketitle

From June to September, West Africa is inundated by a band of monsoon rainfall, originating at the Gulf of Guinea, that reaches a peak in intensity and latitudinal extent in August \parencite{sultan2003west}.
This \emph{West African Monsoon} (\emph{WAM} hereafter) is of immense importance to the inhabitants of the Sahel, the semi-arid belt marking its northernmost zone of influence.
Through the 1970s and 1980s, the region's heavily agricultural economy and society were greatly impacted by ecological and environmental degradation associated with extended periods of drought \parencite{benson1998impact, olsson1993causes, walther2016review}.
These anomalously dry conditions initially provoked concern that a theorised biogeophysical feedback --- in which the rising albedo of a devegetated land surface enhances the sinking of cold, aridifying air \parencite{charney1975dynamics, charney1975drought} --- would drive a rapid southward expansion of the Sahara desert.
Ultimately this runaway desertification did not come to pass; satellite observations instead indicate a widespread \emph{greening} of the Sahel in recent decades \parencite{olsson2005recent, dardel2014re}.

Instinctively, one would attribute this Sahellian revegetation to the tentative recovery in rainfall observed in the region over the same period \parencite{lebel2009recent}.
However, mean precipitation trends conceal the high spatial variability in rainfall patterns as well as a shift toward infrequent, extreme events \parencite{nicholson2013west, panthou2014recent}, while there remain considerable methodological issues in the extraction and interpretation of various vegetation indices from remote sensing data \parencite{fensholt2013assessing, dardel2014rain}.
It is therefore difficult to discern the exact nature of the relationship between the monsoon rainfall and land surface conditions. This is further complicated by the prospect of a highly nonlinear response of vegetation to large scale changes in the WAM.

% Evidence of existence of green Sahara and rapid state change

\printbibliography{}

\end{document}
