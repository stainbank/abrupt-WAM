\documentclass[a4paper]{article}
\usepackage[utf8]{inputenc}
\usepackage[backend=biber, style=nature]{biblatex}
\addbibresource{references.bib}

% Custom commands
\newcommand{\delO}{\ensuremath{\delta ^{18}}O}
\begin{document}

\title{How important is ocean circulation in abrupt change of the West African Monsoon}
\author{Chad Stainbank}
\maketitle

\section{North Africa and the global climate system}
\label{sec:sahelwam}
During a typical August, the West African Monsoon (WAM), having swept from the Gulf of Guinea in June northeastwards across Africa, will reach a peak in intensity over the Sahel, the continent-wide belt of land bordering the Sahara desert that marks the latitudinal limit of its climatic influence \parencite{sultan2003west2, nicholson2013west}.
Given that this semi-arid region has a rapidly growing population that is heavily dependent on agriculture \parencite{sissoko2011agriculture}, the amount and distribution of monsoon rainfall is of immense societal and economic importance.
Indeed, during the 1970s and 1980s, great swathes of the Sahel were afflicted by ecological and environmental degradation associated with extended periods of drought, resulting in widespread crop failures which threatened millions of people with famine \parencite{benson1998impact, olsson1993causes, walther2016review}.
Although initial fears of runaway desertification \parencite{charney1975dynamics, charney1975drought} did not come to pass, the severe long-term impact of this drought underscores the sensitivity of the region to a changing climate.
The latest IPCC Assesment Report \parencite{ipcc2014impacts} outlines the numerous climate vulnerabilities of North Africa, including a high likelihood of extensive ecosystem loss and increasing food and water insecurity, while the Sahel is identified as a climate change "hotspot" \parencite{diffenbaugh2012climate}.
Proactive adaptation to expected rainfall trends is, therefore, a regional priority \parencite{lobell2008prioritizing, sarr2012present}.

Monsoon precipitation, and the vegetation cover it sustains, has a suppressive effect on the quantity of mineral dust emitted into the atmosphere from the Sahara and the Sahel \parencite{brooks2000dust, cowie2013vegetation}.
Terrigenous aerosols act to both absorb and scatter incoming and outgoing radiation \parencite{andreae1995climatic, tegen1996influence, haywood2000estimates}, as well as to enhance and suppress precipitative cloud formation and alter cloud properties, locally and further afield \parencite{rosenfeld2001desert, demott2003african, huang2006possible, karydis2011effect}  
As the dominant global dust source \parencite{tanaka2006numerical}, North Africa plays a major role in determining the Earth's radiation budget, although uncertainty in the magnitude, vertical profile, size distribution and chemical composition of this atmospheric particulate load severely hamper estimates of the size, or even sign, of the net radiative forcing \parencite{claquin1998uncertainties, sokolik2001introduction, durant2009sensitivity}.

The deposition of desert dust over the North Atlantic and Mediterranean provides an important supply of iron, nitrogen and phosphorus to marine ecosystems \parencite{prospero1996saharan, prospero1996atmospheric, guieu2002chemical, mills2004iron, bristow2010fertilizing, okin2011impacts}.
This fertilisation effect is even more significant in the Amazon Basin --- whose carbon sink is in decline \parencite{brienen2015long} --- as Saharan dust is estimated to entirely replenish rainforest soils of the soluble phosphorous continually denuded by rainfall \parencite{swap1992saharan, bristow2010fertilizing, yu2015fertilizing}.
Since this intercontinental nutrient supply is dominated by a few dust source "hotspots" \parencite{koren2006bodele, ben2010transport, schepanski2009saharan, knippertz2010central}, the productivity of nearby oceans and of the Amazon rainforest may be sensitive to even small changes in climate conditions in the Sahara and the Sahel.
With dust an integral feature of the global biogeochemical cycle \parencite{ridgwell2002dust, harrison2001role, jickells2005global, mahowald2005atmospheric}, the North African land surface may be key driver of future change in the climate system.

\section{WAM Variability}
Analyses of recent North African climate trends provide a foundation for the evaluation of the potential response of the WAM to rising CO\textsubscript{2} \parencite{redelsperger2006african}. 
Satellite observations indicate that the Sahel has undergone widespread \emph{greening} in recent decades \parencite{olsson2005recent, dardel2014re}, although there are methodological issues in the extraction and interpretation of vegetation indices from remote sensing data that make land-surface conditions an unreliable indicator of regional climate \parencite{fensholt2013assessing, dardel2014rain}.
Among direct rainfall studies, there remains considerable disagreement as to the extent of recovery from the drought conditions of the late twentieth century \parencite{nicholson2005question, nicholson2013west}.
Furthermore, consideration of the mean Sahellian climate trend conceals the highly spatially discontinuous underlying changes, such as increasing interannual variability and a shift in the Central Sahel toward a more extreme rainfall regime \parencite{nicholson2013west, lebel2009recent, panthou2014recent}.
Since it is not appropriate to simply extrapolate rainfall statistics, a thorough understanding of the mechanisms which govern spatiotemporal variability of the WAM is required to inform North African climate projections \parencite{redelsperger2006african}.

A strong correlation between North African rainfall patterns and global sea surface temperature (SST) anomalies has long been indicated by a range of observational and modelling studies, for which \citeauthor{rodriguez2015variability} \parencite{rodriguez2015variability} provide an excellent overview. % /parencite{folland1986sahel, janowiak1988investigation, rowell1995variability, fontaine1996sea}. 
% ! Warmer tropical Atlantic --> Decreased Sahel rainfall, not increased !
Briefly, a weaker (stronger), more southward (northward) monsoon appears to be driven by colder (warmer) SSTs in the the Mediterranean Sea \parencite{rowell2003impact, gaetani2010influence, fontaine2010impacts}, while converse relationships are found with the tropical eastern Pacific \parencite{rowell2001teleconnections, janicot2001summer, giannini2003oceanic} and Indian \parencite{bader2003impact, giannini2003oceanic} oceans.
% The influence of the tropical Atlantic is more complex...\parencite{lamb1978case, lamb1978large, janowiak1988investigation, ward1998diagnosis} 
Also of importance is the worldwide interhemispheric SST gradient \parencite{folland1986sahel}, with \citeauthor{hoerling2006detection} \parencite{hoerling2006detection} attributing the late twentieth century Sahellian drought primarily to the warming of the tropical South Atlantic relative to tropical North. 
% These patterns are in turn controlled by... e.g. ENSO 

General Circulation Model (GCM) allow the exploration of possible mechanisms of SST influence on North African rainfall.
The interpretation of GCM experiment results is informed largely by the traditional model of WAM dynamics \parencite[see][for comprehensive descriptions]{sultan2000abrupt, sultan2003west2, ramel2006northward}, which posits that the Monsoon flow --- a low level southwesterly flow of marine moisture over the African continent driven by the surface pressure gradient between the Saharan Heat Low (SHL), an intense thermal low that develops during the boreal summer over the hot desert, and the cooler Atlantic Ocean \parencite{hall2006dynamics, grams2010atlantic} --- is precipitated by deep convection at the Intertropical Convergence Zone (ITCZ), the band of trade wind confluence that chases the subsolar point. 
This theory holds that the characteristics of the WAM are determined by the surface moisture flow to \parencite{folland1986sahel, sultan2003west1, hagos2007dynamics}, and strength and latitudinal position of \parencite{janicot1998west, damato1998characteristics, sultan2003west1}, the continental ITCZ --- with which the band of precipitation maxima is often treated synonymously \parencite[e.g.][]{ba1995satellite, ramel2006northward, braconnot2007results, peyrille2016annual}.
Accordingly, \ldots
% Atlantic
% NO LONGER TRUE FOR ATLANTIC: Through the enhancement of the southwesterly and northerly supplies of evaporative moisture, convergence over North Africa is strengthened by warmer SSTs in the Gulf of Guinea \parencite{fontaine2003atmospheric, } and the Mediterranean \parencite{rowell2003impact, gaetani2010influence} respectively.
Primary effect of Mediterranean is to enhance monsoon convergence by increasing the northerly supply of low-level marine-evaporated moisture \parencite{rowell2003impact, gaetani2010influence}.
% Pacific
% Indian

% Therefore, ... (describe SST effect via traditional)
% Warmer tropical Atlantic SSTs --> Northward ITCZ shift. (Ward)

However, \citeauthor{nicholson2009revised} \parencite{nicholson2009revised} presents a divergent interpretation of long-term meteorological reanalysis data.
In this emerging paradigm of WAM dynamics, the tropical rainbelt is largely independent of the aforedescribed surface features \parencite{grist2001study} and is characterised instead by the interaction of zonal jets operating higher in the troposphere \ldots
% : the mid-level African Easterly Jet (AEJ), the high level Tropical Easterly Jet (TEJ) and .
% by a column of ascending air that is bounded to the north by the mid-level African Easterly Jet (AEJ) modulating position, 

% Describe SST effect via new
% These two competing interpretations are both \parencite{lafore2011progress, peyrille2016annual, see post 2009 wam dynamics articles.
% General Circulation Models (GCMs) experiments can be used to determine , and analyses of results from the Coupled Model Intercomparison Project Phase 5 (CMIP5) 

% ~~~~OLD~~~~~
% Therefore, in order to make useful projections of the response of the Sahellian landscape to future changes in global climate, it is crucial to develop a better understanding of the mechanisms which govern the timing, strength and position of the monsoon over North Africa. 
% 
% \citeauthor{nicholson2013west} \parencite{nicholson2013west} provide a comprehensive overview of the wide range of factors theorised to modulate WAM variability at interannual to interdecadal timescales.
% For example, several General Circulation Model (GCM) experiments have succesfully reproduced much of the sensitivity of monsoon rainfall to global sea surface temperatures apparent in the observational record \parencite[e.g.][]{rowell1995variability, giannini2003oceanic, lu2005oceanic}, while others demonstrate the importance of land-surface--atmosphere interactions in amplifying these oceanic forcings \parencite{giannini2005dynamics, kucharski2013further}.
% However, no single dominant influence on the WAM has emerged as GCMs present an array of contradictory mechanisms of monsoon control, particularly with regard to the relative influence of each ocean basin... 
% As a result, there is no agreement with regard to the climatological future of the Sahel \parencite{druyan2011studies}.
% % In addition, future changes are huge compared to decadal variability.
% It is therefore useful to look at the palaeoenvironmental record in order to inform projections of how the North African landscape may respond to future changes in global climate.
% % This is further complicated by the prospect of a threshold-like response to large scale changes in the WAM indicated in the North African palaeoenvironmental record.
% 
% % Evidence of existence of green Sahara and rapid state change
% 
% There are a wide range of proxy data indicating that the global climate was highly arid during the Last Glacial Maximum \parencite{harrison2001role}, including peaks in the rate of aeolian dust deposition recorded in cores of polar ice \parencite{petit1990palaeoclimatological} and marine sediment \parencite{tiedemann1989climatic, rea1994paleoclimatic}. 
% However, thick layers of pollen-laden lacustrine sediments in the Sahara desert attest to the region's transition, over the subsequent deglaciation, to humid conditions that could support extensive and highly productive wetland ecosystems \parencite{ritchie1985sediment, lezine1990across}.
% On the assumption that the balance between water input and evaporation controls the oxygen isotopic signal in lake sediments, \citeauthor{gasse1990arid} \parencite{gasse1990arid} find that high water levels succeeded arid conditions in North Africa from the $\sim$15 ka termination Heinrich Stadial to $\sim$5 ka, with a brief positive \delO{} excursion signalling the resumption of aridity during the Younger Dryas stadial (YD).
% In addition, satellite radar imagery has been used to identify the geomorphological imprint of immense Saharan lakes which attest to the magnitude of this \emph{African Humid Period} (AHP) \parencite{schuster2005holocene, drake2006shorelines}.
% Although highly local influences such as topography and groundwater availability can complicate extrapolation to a regional palaeclimate \parencite{baumhauer1991palaeolakes}, these proxy data are generally accepted to provide strong evidence for an extreme northward incursion of the WAM into North Africa during the mid-Holocene.
% 
% The AHP was not a unique occurence: using marine sediment's magnetic properties \parencite{bloemendal1989evidence, larrasoana2003three}non-carbonate fraction \parencite{tiedemann1994astronomic} as a proxy for the Saharan dust flux, aridity in North Africa can be shown to have been paced by various components of the Milankovitch cycles over the past 5 Ma.
% However, despite the gradualness of the orbital forcing, sharp discontinuities in very high resolution Atlantic palaeosedimentation records indicate that the most recent arid--humid transitions --- at the end of the LGM, into and out of the YD, and at the end of AHP --- were exceptionally abrupt \parencite{demenocal2000abrupt}.
% The assertion of century-scale climatological transitions is not without criticism.
% % For example... is it just wind?
% % In addition, general circulation models (GCMs) featuring atmosphere have largely failed to reproduce a threshold-like response
% % critcise evidence of abruptness - e.g. is it not just wind?
% % show other evidence of abruptness
% % Assuming evidence is legit, shows that there is some nonlinearity in the system
% % Obvious candidate is monsoon
% 

\printbibliography{style=nature}

\end{document}
