\documentclass[a4paper]{article}
\usepackage[utf8]{inputenc}
\usepackage[backend=biber, style=authoryear]{biblatex}
\addbibresource{references.bib}
\begin{document}

\title{How important is ocean circulation in abrupt change of the West African Monsoon}
\author{Chad Stainbank}
\maketitle

From June to September, West Africa is inundated by a band of monsoon rainfall, originating at the Gulf of Guinea, that reaches a peak in intensity and latitudinal extent in August \parencite{sultan2003west}.
This \emph{West African Monsoon} (hereafter WAM) is of immense importance to the inhabitants of the Sahel, the semi-arid belt marking its northernmost zone of influence.
Through the 1970s and 1980s, the region's heavily agricultural economy and society were greatly impacted by ecological and environmental degradation associated with extended periods of drought \parencite{benson1998impact, olsson1993causes, walther2016review}.
These anomalously dry conditions initially provoked concern that a theorised biogeophysical feedback --- in which a rise in the land surface albedo enhances the sinking of dry over an area, further worsening the initial devegetation \parencite{charney1975dynamics, charney1975drought} --- would drive a rapid southward expansion of the Sahara desert.
Thankfully this runaway desertification did not come to pass; satellite observations instead indicate a widespread \emph{greening} of the Sahel in recent decades \parencite{olsson2005recent, dardel2014re}.

However, it is not appropriate to entirely attribute modern revegetation to the tentative recovery in Sahellian rainfall observed over the same period \parencite{lebel2009recent}.
Overall precipitation trends are complicated by increasing spatial variability and a shift in rainfall patterns toward infrequent extreme events \parencite{nicholson2013west, panthou2014recent}, while there are considerable methodological issues in the extraction and interpretation of various vegetation indices from remote sensing data \parencite{fensholt2013assessing, dardel2014rain}.
It is therefore difficult to discern the nature of the relationship between monsoon changes and land surface conditions, which is further complicated by the possibility of nonlinear, threshold behaviour. 

% Nevertheless, given the importance of any potential changes, it is crucial to understand the response of the WAM to climate change


\printbibliography{}

\end{document}
